\section{Keras APIs to build DL models}
Follow this link: https://machinelearningmastery.com/keras-functional-api-deep-learning/
https://www.activestate.com/resources/quick-reads/what-is-a-keras-model/

An API (Application Programming Interface) is a set of tools and functions that allow you to interact with a software or framework. In simple terms, think of an API as a menu at a restaurant: it tells you what dishes (functions) you can order (use), without needing to know how they are prepared (the underlying implementation). In the context of Keras, APIs let you define, build, and train deep learning models easily, without knowing much about how the Keras functions are built.

Types of Keras APIs for Building Deep Learning Models
Keras offers two main APIs for building models:

Sequential API: A simple, linear stack of layers where each layer has one input and one output.
Functional API: A more flexible approach where you can define complex architectures like multi-input, multi-output models, or models with shared layers.
\subsection{Sequential API}
The Sequential API is the easiest and most intuitive way to build a model. You stack layers one after another, like building blocks. It works well when your model has a straightforward flow of data from input to output without branching.

Why is it popular?
Simplicity: Perfect for beginners or for straightforward tasks.
Fast to implement: Requires minimal coding effort.
Readable: Easy to understand and debug.
Example in Rainfall-Runoff Modeling
Suppose you want to predict river runoff based on rainfall data. Using the Sequential API, you can create a simple feedforward neural network where data flows through layers in one direction:



\subsection{Functional API}
The Functional API is more flexible and allows you to define complex architectures. Unlike the Sequential API, you can connect layers in arbitrary ways, such as branching or merging data flows.

Why do hydrologists need it?
Complex Models: You may need models with multiple inputs (e.g., rainfall from different regions and soil moisture data) or multiple outputs (e.g., runoff and evaporation).
Shared Layers: Useful when you want the same computation applied to different inputs, such as rainfall data from multiple time steps.
Custom Architectures: Models with residual connections (like in ResNet) or attention mechanisms often require the Functional API.
Example in Rainfall-Runoff Modeling
Let’s say you want to predict runoff using rainfall data from two separate regions (Region A and Region B). Each region’s data requires a different processing branch before being combined for the final prediction:

Here, rainfall data from Region A and Region B are processed separately but combined to predict runoff. This architecture is only possible with the Functional API. 