\chapter*{About the Book}

Also, most of the deep learning books available in the market are focused on delivering deep learning concepts based on natural language processing and image analysis. However, the water and earth system science domains strongly require a book that focuses on the time series problems because as earth or climate or water scientists we deal with time series data across many scenarios. Additionally, we require understanding and processing the image data to get insights from the remote sensing products for topics related to semantic segmentation, LULC classification, and anomaly detection problems. 
There are a few books that have been recently published to give the readers the essence of deep learning. Although they have explained the theories, however, these books are full of unnecessary mathematical equations and devoid of intuitions. Additionally, there are no coding examples to demonstrate that this is a great book. Mostly, they focus on a literature review and developing a very simple and rudimentary understanding of deep learning. However, we provide in-depth coding examples with real-world data in detail. There are several and a variety of examples presented. 


This book is designed for anyone passionate about applying deep learning techniques to hydrology, water resources, climate science, geology, earth sciences, remote sensing, and related domains. Whether you’re a researcher, student, or professional, this book will provide immense value by exploring how deep learning can address real-world problems in these fields.

This is a practical, hands-on guide to deep learning that avoids overly complex mathematical notation. Instead, it emphasizes intuition and understanding through clear explanations and step-by-step Python code examples. By following this book, you’ll learn to build deep learning models using the \textit{TensorFlow} and \textit{Keras} frameworks, with a focus on solving domain-specific challenges. Even if you prefer \textit{PyTorch}, the foundational knowledge and practical applications in this book will serve as a robust resource for tackling similar problems.

\section*{Real-World Problems Explored in This Book}
The book is packed with examples drawn from real-world hydrological and environmental challenges, such as:
\begin{enumerate}
    \item Drought classification.
    \item Rainfall-runoff modeling.
    \item Hourly temperature prediction.
    \item Reservoir water level prediction/classification.
    \item Land Use Land Cover (LULC) classification.
    \item Soil moisture prediction.
    \item Heatwave classification.
    \item Digital Elevation Model (DEM) and LiDAR data processing.
    \item Vegetation health detection.
    \item Drought prediction and monitoring in real time.
\end{enumerate}

\section*{Who Should Read This Book?}
This book is for:
\begin{itemize}
    \item \textbf{Hydrologists, climate scientists, geologists, and remote sensing professionals} who want to learn how to apply deep learning techniques to their data-driven problems.
    \item \textbf{Students and researchers} looking for an intuitive and accessible introduction to deep learning concepts and applications in environmental and earth sciences.
    \item \textbf{Data scientists and machine learning practitioners} who want to expand their expertise into domain-specific problems, such as water resource management and climate modeling.
    \item \textbf{Professionals transitioning to deep learning} and seeking practical examples to get started with TensorFlow and Keras.
\end{itemize}

No advanced mathematics background is required to understand this book; high school–level mathematics is sufficient. Readers should, however, have basic proficiency in Python programming and familiarity with libraries like NumPy. For those new to machine learning or deep learning, this book introduces the necessary foundations from scratch.

\section*{What You'll Learn}
After reading this book, you’ll:
\begin{itemize}
    \item Have a solid understanding of deep learning concepts and their applications in hydrology, climate science, and related domains.
    \item Be able to use TensorFlow and Keras to build, train, and evaluate deep learning models for diverse real-world problems.
    \item Understand key workflows, best practices, and challenges in applying deep learning to hydrological and environmental data.
    \item Gain practical skills in solving problems involving time series, spatial data, and remote sensing imagery.
\end{itemize}

Even if you are using PyTorch, this book will still serve as an exceptional resource to understand the concepts and workflows required for deep learning in environmental sciences. The examples provided can be adapted to your preferred framework, making this book a valuable asset regardless of your tools of choice.

\begin{mdframed}[linewidth=1pt, linecolor=gray, backgroundcolor=gray!5]
\textbf{Note:} While this book focuses on TensorFlow and Keras, the problem-solving strategies and examples are framework-agnostic. Readers using PyTorch or other deep learning libraries can still gain immense insights and apply the concepts discussed here effectively.
\end{mdframed}

This book offers not just a theoretical understanding of deep learning but also equips you with practical tools to address some of the most pressing challenges in hydrology, water resources, and environmental sciences.
